\documentclass{article}

\usepackage{amsmath}
\usepackage{amsthm}

\newtheorem{theorem}{Theorem}

\title{CS325 Project 1 Report}
\author{Erika Crowe and Dean Johnson}

\begin{document}
\maketitle


\subsection*{Run-time Analysis}


\emph{Steps after each algorithm title are placeholders.\newline Add appropriate steps after each algorithm title.}

\begin{tabbing}
  {\sc Algorithm 1: Enumeration}\\
  \qquad \= $x = 0$ \\
  \> for $i = 1, \ldots, n$\\
  \> \qquad \= $x = x+i$\\
  \> return $x$
\end{tabbing}

\begin{tabbing}
  {\sc Algorithm 2: Better Enumeration}\\
  \qquad \= $x = 0$ \\
  \> for $i = 1, \ldots, n$\\
  \> \qquad \= $x = x+i$\\
  \> return $x$
\end{tabbing}

\begin{tabbing}
  {\sc Algorithm 3: Even Better Enumeration}\\
  \qquad \= $x = 0$ \\
  \> for $i = 1, \ldots, n$\\
  \> \qquad \= $x = x+i$\\
  \> return $x$
\end{tabbing}

\subsection*{Correctnes}

\begin{theorem}
  If \emph{$\{y_{j1},y_{j2},...,y_{jt}\}$} is the visible subset of \emph{$\{y_{1},y_{2},...,y_{i-1}\}$} where \emph{$(t \geq i-1)$} then $\{y_{j1},y_{j2},...,y_{jk},y_{i}\}$ is the visible subset of $\{y_{1},y_{2},...,y_{i}\}$ where $y_{jk}$ is the last line such that $y_{jk}(x^{*}) \geq y_{i}(x^{*})$ where $(x^{*},y_{jk}(x^{*}))$ is the point of intersection of the lines $y_{jk}$ and $y_{jk-1}$.
\end{theorem}

\begin{proof}
prove the theorem here
\end{proof}


Formulae can be set within a line such as this: $f(x,y) = (x+y)(x-y)$
or set out as in Equation~(\ref{eq:1}).
\begin{equation}
  \label{eq:1}
  f(x,y) = (x+y)(x-y)
\end{equation}
Here is an unnumbered multi-line formula:
\begin{eqnarray*}
  \label{eq:2}
  f(x,y) &= & (x+y)(x-y) \\
  & = & x(x-y) + y(x-y) \\
  & = & x^2 - xy + xy - y^2 \\
  & = & x^2-y^2
\end{eqnarray*}

\subsection*{Experimental Correctness}


\subsection*{Experimental Analysis}


\subsection*{Extrapolation and Interpolation}


\end{document}
